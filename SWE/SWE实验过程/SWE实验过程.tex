\documentclass{article}
\usepackage[a4paper,innermargin=1.2in,outermargin=1.2in,
bottom=1.5in,marginparwidth=1in,marginparsep
=3mm]{geometry}
\usepackage{amsmath,amsthm,amssymb,enumerate}
\usepackage{ctex}
\usepackage{natbib}
\usepackage{graphicx}
\graphicspath{{images/}}
\usepackage{float}
\usepackage{subfigure}
\usepackage{color}
\usepackage{array}
\title{浅水方程}
\author{MG21210021李庆春}
%\date{2022.8.20}
\linespread{1.25}
\bibliographystyle{plain}

\usepackage{listings}
\lstset{
	columns=fixed,       
	numbers=left,                                        % 在左侧显示行号
	numberstyle=\tiny\color{black},                       % 设定行号格式
	frame=none,                                          % 不显示背景边框
	backgroundcolor=\color[RGB]{245,245,244},            % 设定背景颜色
	keywordstyle=\color[RGB]{40,40,255},                 % 设定关键字颜色
	numberstyle=\footnotesize\color{black},           
	commentstyle=\it\color[RGB]{0,96,96},                % 设置代码注释的格式
	stringstyle=\rmfamily\slshape\color[RGB]{128,0,0},   % 设置字符串格式
	showstringspaces=false,                              % 不显示字符串中的空格
	language=python,                                        % 设置语言
}


\begin{document}
\maketitle
\section{2D浅水方程简述}
\begin{equation}
	\begin{cases}
		\frac{du}{dt} - fv = -g \frac{d\eta}{dx} + \frac{\tau_x}{\rho_0 H} - \kappa u  \\
		\frac{dv}{dt} + fv = -g \frac{d\eta}{dy} + \frac{\tau_y}{\rho_0 H} - \kappa v  \\
		\frac{d\eta}{dt} + \frac{d(\eta + H)u}{dx} + \frac{d(\eta + H)v}{dy}= \sigma - w
		\label{swe}
	\end{cases}
\end{equation}
以上是二维浅水方程表达式,其中动量方程是线性的,连续方程是非线性的,自变量x,y,t的含义是显然的,应变量u是水平方向的流速,v是垂直方向的流速,$\eta$是水面动态海拔;其中$f = f_0 + \beta y$是全维度变换的科里奥利斯参数;$\kappa$是跟摩擦相关的系数;$\tau_x \tau_y$是跟风应力相关的系数;$\sigma$是跟质量源相关的系数;$w$是跟质量汇相关的系数;先不考虑这些额外的系数,将方程\ref{swe}简化为: \\
\begin{equation}
	\begin{cases}
		\frac{du}{dt} = -g \frac{d\eta}{dx} \\
		\frac{dv}{dt} = -g \frac{d\eta}{dy} \\
		\frac{d\eta}{dt} = -\frac{d(\eta + H)u}{dx} - \frac{d(\eta + H)v}{dy}
	\end{cases}
\end{equation}
等价于
\begin{equation}
	\begin{cases}
		\frac{du}{dt} = -g \frac{d\eta}{dx} \\
		\frac{dv}{dt} = -g \frac{d\eta}{dy} \\
		\frac{d\eta}{dt} = - u \frac{d\eta}{dx} - v \frac{d\eta}{dy}  - (\eta + H )(\frac{du}{dx} + \frac{dv}{dy})
	\end{cases}
\end{equation}
该方程的CFL条件为:
\begin{equation}
	\begin{cases}
		dt \leq \frac{min(dx,dy)}{\sqrt{gH}}  \\
		\alpha \ll 1 \qquad (if \quad coriolis \quad is \quad used)
	\end{cases}
\end{equation}
其中dx,dy是网格间距,g是重力加速度,H是静水深度。\\
下面是把代码贴给GPT,反向翻译出来的公式,与上面是吻合的,所以代码没问题
\begin{equation}
\begin{cases}
	\frac{\partial u}{\partial t} = -g \frac{\partial \eta}{\partial x} \\
	\frac{\partial v}{\partial t} = -g \frac{\partial \eta}{\partial y} \\
	\frac{\partial \eta}{\partial t} = -\frac{\partial}{\partial x}\left((\eta + H)u\right) - \frac{\partial}{\partial y}\left((\eta + H)v\right)
\end{cases}
\end{equation}

\section{1D浅水方程简述}
\begin{equation}
	\begin{cases}
		\frac{\partial}{\partial t} h + \frac{\partial}{\partial x} hv = 0  \\
		\frac{\partial}{\partial t} hv + \frac{\partial}{\partial x} (hv^2 + \frac{gh^2}{2}) = 0
		\label{swe-1D}
	\end{cases}
\end{equation}
或
\begin{equation}
	\begin{cases}
		\frac{\partial h}{\partial t} = - h\frac{\partial v}{\partial x} - v\frac{\partial h}{\partial x} \\
		h\frac{\partial v}{\partial t} + v\frac{\partial h}{\partial t} = - 2hv\frac{\partial v}{\partial x} - (v^2 + gh)\frac{\partial h}{\partial x}
	\end{cases}
\end{equation}
第一个方程式质量守恒方程,第二个方程是动量守恒方程;对应的积分形式如下:
\begin{equation}
	\begin{cases}
		\frac{d}{dt} \int^{x_1}_{x_0} h dx = h(x_0, t) v(x_0, t) -  h(x_1, t) v(x_1, t) = \int^{x_1}_{x_0} -\frac{\partial}{\partial x} (hv)dx \\
		\frac{d}{dt} \int^{x_1}_{x_0} hv dx = h(x_0, t)v^2(x_0, t) + \frac{gh^2(x_0, t)}{2} - h(x_1, t)v^2(x_1, t) - \frac{gh^2(x_1, t)}{2}
	\end{cases}
\end{equation}
对于浅水方程,对于每个区间$[x_i,x_{i+1}]$,需要存储质量的平均值和动量的平均值:
\begin{equation}
	\begin{cases}
		H_i = \frac{1}{\Delta x} \int^{x_{i+1}}_{x_i} hdx \\
		M_i = \frac{1}{\Delta x} \int^{x_{i+1}}_{x_i} hvdx
	\end{cases}
\end{equation}
把$H_i, M_i$带入积分形式:
\begin{equation}
	\begin{cases}
		\frac{d}{dt} H_i \Delta x = h(x_i, t) v(x_i, t) -  h(x_{i+1}, t) v(x_{i+1}, t) \\
		\frac{d}{dt} M_i \Delta x = h(x_i, t)v^2(x_i, t) - h(x_{i+1}, t)v^2(x_{i+1}, t) + \frac{gh^2(x_i, t)}{2} - \frac{gh^2(x_{i+1}, t)}{2}
	\end{cases}
\end{equation}
对于任意的$x \in (x_i, x_{i+1})$,我们可以把$h(x, t)$和$v(x, t)$ 看成常值函数,因为:
\begin{equation}
	\begin{cases}
		h(x, t) \approx H_i(t) \\
		h(x,t)v(x,t) \approx M_i(t) \\
		v(x,t) \approx \frac{M_i(t)}{H_i(t)}
	\end{cases}
\end{equation}
在边界点$x_i$和$x_{i+1}$上,我们可以通过对边界点左右两个区间的守恒量取平均值,对于FVM以及其他同类数值解函数存在不连续性的数值方法,这样给定两个相邻的控制体积上的函数解,求两个控制体积交界处的函数值的策略被称为数值通量(numerical flux)。这样简单取两个值的平均的通量被称为中央通量(central flux),不过它只是众多通量选择中的一种。在方程比较简单并且解函数比较平滑的时候,中央通量还是比较可靠的。但是对于SWE,中央通量是数值不稳定(numerically unstable)的。为了解决这一问题,这里我们采用Lax–Friedrichs通量。Lax-Friedrich通量很容易计算:分别取左边和右边两区间上的值计算通量,取二值的平均,再加上一修正参数,即系统中的波速乘以该守恒量在两个区间上的差再除以二。这一修正参数被称为数值消散(numerical dissipation)。\\
\begin{equation}
	h(x_i, t) v(x_i, t) \approx \frac{M_{i-1} + M_i}{2} + wavespeed \cdot \frac{H_{i-1} - H_i}{2}
\end{equation}
对于边界点$x_0$和$x_n$,当我们使用反射强边界条件时,我们可以人为地设置两个假想体积(ghost cell)。比如,我们可以认为在区间$[x_0,x_1]$的左边还有一个区间: $[x_{-1},x_0]$。并且$H_{-1} = H_0, M_{-1} = -M_0$。也就是说:我们认为在$[x_{-1},x_0]$上的水和$[x_0,x_1]$上的水有相同的高度但是相反的运动方向。形象一点来说的话,那就是我们可以认为$x_0$上有一面镜子,只不过这个“镜子”反射的不是光,而是动量。\\
所以,我们得到最终的FVM格式:
\begin{equation}
	\begin{cases}
		\frac{d}{dt} H_i(t) = \frac{1}{\Delta x} [ h(x_i, t) v(x_i, t) -  h(x_{i+1}, t) v(x_{i+1}, t) ] \\
		\frac{d}{dt} M_i(t) = \frac{1}{\Delta x} \{[h(x_i, t)v^2(x_i, t) + \frac{gh^2(x_i, t)}{2} ] - [h(x_{i+1}, t)v^2(x_{i+1}, t) + \frac{gh^2(x_{i+1}, t)}{2}]\}
	\end{cases}
\end{equation}
其中
\begin{equation}
	\begin{cases}
		h(x_i, t) v(x_i, t) = \frac{M_{i-1} + M_i}{2} + c \cdot \frac{H_{i-1} - H_i}{2} \\
		h(x_i, t)v^2(x_i, t) + \frac{gh^2(x_i, t)}{2}  = \frac{\frac{M^2_{i-1}}{H_{i-1}} + \frac{M^2_{i}}{H_{i}}}{2} + \frac{gH^2_{i-1} + gH^2_i}{4} +c \cdot \frac{M_{i-1} - M_i}{2} 
	\end{cases}
\end{equation}
其中c为波速,$H_{-1} = H_0, H_{n+1} = H_n,  M_{-1} = -M_0, M_{n+1} = -M_n$ \\ 
CFL条件:$c \cdot \Delta t < \Delta x$ \\
所以最终格式为:
\begin{equation}
	\begin{cases}
		\frac{d}{dt} H_i(t) = \frac{1}{2\Delta x} [M_{i-1} - M_{i+1} + c \cdot (H_{i-1} -2H_i + H_{i+1})] \\
		\frac{d}{dt} M_i(t) = \frac{1}{2\Delta x} [(\frac{M^2_{i-1}}{H_{i-1}} - \frac{M^2_{i+1}}{H_{i+1}}) + \frac{g}{2}(H^2_{i-1} - H^2_{i+1}) + c(M_{i-1}-2M_i+M_{i+1})]
	\end{cases}
\end{equation}

\end{document}