\documentclass{article}
\usepackage[a4paper,innermargin=1.2in,outermargin=1.2in,
bottom=1.5in,marginparwidth=1in,marginparsep
=3mm]{geometry}
\usepackage{amsmath,amsthm,amssymb,enumerate}
\usepackage{ctex}
\usepackage{natbib}
\usepackage{graphicx}
\graphicspath{{images/}}
\usepackage{float}
\usepackage{subfigure}
\usepackage{color}
\usepackage{array}
\title{浅水方程}
\author{MG21210021李庆春}
%\date{2022.8.20}
\linespread{1.25}
\bibliographystyle{plain}

\usepackage{listings}
\lstset{
	columns=fixed,       
	numbers=left,                                        % 在左侧显示行号
	numberstyle=\tiny\color{black},                       % 设定行号格式
	frame=none,                                          % 不显示背景边框
	backgroundcolor=\color[RGB]{245,245,244},            % 设定背景颜色
	keywordstyle=\color[RGB]{40,40,255},                 % 设定关键字颜色
	numberstyle=\footnotesize\color{black},           
	commentstyle=\it\color[RGB]{0,96,96},                % 设置代码注释的格式
	stringstyle=\rmfamily\slshape\color[RGB]{128,0,0},   % 设置字符串格式
	showstringspaces=false,                              % 不显示字符串中的空格
	language=python,                                        % 设置语言
}


\begin{document}
\maketitle
\section{浅水方程简述}
\begin{equation}
	\begin{cases}
		\frac{du}{dt} - fv = -g \frac{d\eta}{dx} + \frac{\tau_x}{\rho_0 H} - \kappa u  \\
		\frac{dv}{dt} + fv = -g \frac{d\eta}{dy} + \frac{\tau_y}{\rho_0 H} - \kappa v  \\
		\frac{d\eta}{dt} + \frac{d(\eta + H)u}{dx} + \frac{d(\eta + H)u}{dy}= \sigma - w
		\label{swe}
	\end{cases}
\end{equation}
以上是二维浅水方程表达式,其中动量方程是线性的,连续方程是非线性的,自变量x,y,t的含义是显然的,应变量u是水平方向的流速,v是垂直方向的流速,$\eta$是水面动态海拔;其中$f = f_0 + \beta y$是全维度变换的科里奥利斯参数;$\kappa$是跟摩擦相关的系数;$\tau_x \tau_y$是跟风应力相关的系数;$\sigma$是跟质量源相关的系数;$w$是跟质量汇相关的系数;先不考虑这些额外的系数,将方程\ref{swe}简化为: \\
\begin{equation}
	\begin{cases}
		\frac{du}{dt} = -g \frac{d\eta}{dx} \\
		\frac{dv}{dt} = -g \frac{d\eta}{dy} \\
		\frac{d\eta}{dt} + \frac{d(\eta + H)u}{dx} + \frac{d(\eta + H)u}{dy}= 0
	\end{cases}
\end{equation}
该方程的CFL条件为:
\begin{equation}
	\begin{cases}
		dt \leq \frac{min(dx,dy)}{\sqrt{gH}}  \\
		\alpha \ll 1 \qquad (if \quad coriolis \quad is \quad used)
	\end{cases}
\end{equation}
其中dx,dy是网格间距,g是重力加速度,H是静水深度。

\end{document}