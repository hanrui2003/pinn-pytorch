\documentclass{article}
\usepackage[a4paper,innermargin=1.2in,outermargin=1.2in,
bottom=1.5in,marginparwidth=1in,marginparsep
=3mm]{geometry}
\usepackage{amsmath,amsthm,amssymb,enumerate}
\usepackage{ctex}
\usepackage{natbib}
\usepackage{graphicx}
\graphicspath{{images/}}
\usepackage{float}
\usepackage{subfigure}
\usepackage{color}
\usepackage{array}
\title{实验结果梳理}
\author{MG21210021李庆春}
%\date{2022.8.20}
\linespread{1.25}
\bibliographystyle{plain}

\usepackage{listings}
\lstset{
	columns=fixed,       
	numbers=left,                                        % 在左侧显示行号
	numberstyle=\tiny\color{black},                       % 设定行号格式
	frame=none,                                          % 不显示背景边框
	backgroundcolor=\color[RGB]{245,245,244},            % 设定背景颜色
	keywordstyle=\color[RGB]{40,40,255},                 % 设定关键字颜色
	numberstyle=\footnotesize\color{black},           
	commentstyle=\it\color[RGB]{0,96,96},                % 设置代码注释的格式
	stringstyle=\rmfamily\slshape\color[RGB]{128,0,0},   % 设置字符串格式
	showstringspaces=false,                              % 不显示字符串中的空格
	language=python,                                        % 设置语言
}


\begin{document}
\maketitle

\section{反应扩散方程简述}
\begin{equation}
	\frac{\partial u}{\partial t}=D\frac{\partial^2 u}{\partial x^2}+ku^2 \qquad (x,t) \in (0,1] \times(0,1]
\end{equation}
其中,扩散系数D=0.01,反应速率k=0.01,边值为零函数;\\

\subsection{DeepONet}
考虑初值为方程参数的问题,即初值不确定,记为$u_0(x)$,由高斯径向基函数生成5000个初值函数:\\
\begin{equation}
	k(x,x') = \sigma^2 exp(-\frac{||x-x'||^2}{2l^2}) \label{eq:rbf}
\end{equation}
其中,$\sigma=1, l=0.2$ \\
分支网络的结构为:[100, 64, 64, 64, 64, 64],其输入由5000个初值函数构成;\\
主干网络的结构为:[2, 64, 64, 64, 64, 64],主干网络的输入训练点由三部分构成:
\begin{enumerate}
	\item 100个初值点;
	\item 200个边值点,每个边100个;
	\item 200个配置单,即物理信息点;
\end{enumerate}
训练结果为:相对误差大概在$10^{-3} \sim 10^{-2}$,图\ref{adr_ic_2d}和图\ref{adr_ic_3d}为一个测试样例:
\begin{figure}[h]
	\centering
	\subfigure[数值解]{
		\includegraphics[width=0.45\linewidth]{adr_ic_real_2d}
	}
	\subfigure[神经网络解]{
		\includegraphics[width=0.45\linewidth]{adr_ic_nn_2d}
	}
	\caption{等高图}
	\label{adr_ic_2d}
\end{figure}
\begin{figure}[h]
	\centering
	\subfigure[数值解]{
		\includegraphics[width=0.45\linewidth]{adr_ic_real_3d}
	}
	\subfigure[神经网络解]{
		\includegraphics[width=0.45\linewidth]{adr_ic_nn_3d}
	}
	\caption{三维图}
	\label{adr_ic_3d}
\end{figure}

\section{1D浅水方程简述}
\begin{equation}
	\begin{cases}
		\frac{\partial}{\partial t} h + \frac{\partial}{\partial x} hv = 0  \\
		\frac{\partial}{\partial t} hv + \frac{\partial}{\partial x} (hv^2 + \frac{gh^2}{2}) = 0
		\label{swe-1D}
	\end{cases}
\end{equation}
或
\begin{equation}
	\begin{cases}
		\frac{\partial h}{\partial t} = - h\frac{\partial v}{\partial x} - v\frac{\partial h}{\partial x} \\
		h\frac{\partial v}{\partial t} + v\frac{\partial h}{\partial t} = - 2hv\frac{\partial v}{\partial x} - (v^2 + gh)\frac{\partial h}{\partial x}
	\end{cases}
\end{equation}

\subsection{PINN}
边值条件为$v=0$,初值为高斯径向基\ref*{eq:rbf}生成的随机函数,此时$\sigma=1, l=0.05$,观测值为$x=0.5, t=[.1, .2, .3, .4, .5, .6, .7, .8, .9, 1.]$上的数据,配置点(物理信息点)为10000个,网络架构为[2, 32, 32, 32, 32, 32, 32, 2]。训练结果的相对误差大概在$10^{-3}$,图\ref{swe_1d_pinn_obs}为两个时间片的数据,其中红色实线为神经网络解:
\begin{figure}[h]
	\centering
	\subfigure[t=0.1]{
		\includegraphics[width=0.45\linewidth]{swe_1d_pinn_obs_1}
	}
	\subfigure[t=0.8]{
		\includegraphics[width=0.45\linewidth]{swe_1d_pinn_obs_2}
	}
	\caption{SWE-PINN}
	\label{swe_1d_pinn_obs}
\end{figure}

\subsection{DeepONet}
考虑初值为方程参数的问题,即初值不确定,记为$u_0(x)$,由高斯函数生成:\\
\begin{equation}
	u_0(x) = 0.1+0.1 \times \exp(-64(x-\mu)^2)
\end{equation}
其中,$\mu \in [0.0,0.1,0.2,\dots,0.9,1.0]$ \\
分支网络的结构为:[132, 128, 64, 64, 64, 64],其输入由上述高斯函数构成;\\
主干网络的结构为:[2, 64, 64, 64, 64, 64],主干网络的输入训练点由两部分构成:
\begin{enumerate}
	\item 100个初值点;
	\item 5000个配置单,即物理信息点;
\end{enumerate}
训练结果的相对误差大概在$10^{-3}$,图\ref{swe_1d_don_obs}为两个时间片的数据,其中红色实线为神经网络解:
\begin{figure}[h]
	\centering
	\subfigure[t=0.1]{
		\includegraphics[width=0.45\linewidth]{swe_1d_don_obs_1}
	}
	\subfigure[t=0.5]{
		\includegraphics[width=0.45\linewidth]{swe_1d_don_obs_2}
	}
	\caption{SWE-DeepONet}
	\label{swe_1d_don_obs}
\end{figure}

\end{document}