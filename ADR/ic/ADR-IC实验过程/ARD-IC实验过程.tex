\documentclass{article}
\usepackage[a4paper,innermargin=1.2in,outermargin=1.2in,
bottom=1.5in,marginparwidth=1in,marginparsep
=3mm]{geometry}
\usepackage{amsmath,amsthm,amssymb,enumerate}
\usepackage{ctex}
\usepackage{natbib}
\usepackage{graphicx}
\graphicspath{{images/}}
\usepackage{float}
\usepackage{subfigure}
\usepackage{color}
\usepackage{array}
\title{反应扩散方程}
\author{MG21210021李庆春}
%\date{2022.8.20}
\linespread{1.25}
\bibliographystyle{plain}

\usepackage{listings}
\lstset{
	columns=fixed,       
	numbers=left,                                        % 在左侧显示行号
	numberstyle=\tiny\color{black},                       % 设定行号格式
	frame=none,                                          % 不显示背景边框
	backgroundcolor=\color[RGB]{245,245,244},            % 设定背景颜色
	keywordstyle=\color[RGB]{40,40,255},                 % 设定关键字颜色
	numberstyle=\footnotesize\color{black},           
	commentstyle=\it\color[RGB]{0,96,96},                % 设置代码注释的格式
	stringstyle=\rmfamily\slshape\color[RGB]{128,0,0},   % 设置字符串格式
	showstringspaces=false,                              % 不显示字符串中的空格
	language=python,                                        % 设置语言
}


\begin{document}
	
\maketitle
	
\section{反应扩散方程简述}
\begin{equation}
	\frac{\partial u}{\partial t}=D\frac{\partial^2 u}{\partial x^2}+ku^2 \qquad (x,t) \in (0,1] \times(0,1]
\end{equation}
其中,扩散系数D=0.01,反应速率k=0.01,边值为零函数;\\
考虑初值为方程参数的问题,即初值不确定,记为$u_0(x)$;\\
我们采用DeepONet去拟合这个算子,记$G_{\theta}$为神经网络算子,$G$为真解算子,则对于任意给的初值函数$u^{(i)}_0$:
\begin{equation}
	G_{\theta}(u^{(i)}_0)(x,t)\approx G(u^{(i)}_0)(x,t)= u^{(i)}(x,t)
\end{equation}
\begin{equation}
	\frac{\partial G_{\theta}(u^{(i)}_0)(x,t)}{\partial t} \approx D\frac{\partial^2 G_{\theta}(u^{(i)}_0)(x,t)}{\partial x^2} + k[G_{\theta}(u^{(i)}_0)(x,t)]^2
\end{equation}
所以我们可以构造两类损失,第一类是神经网络模型的损失,即对于带有label的数据(也就是本例中满足初边值条件的点),它们的label值和网络预测值之间的差异,我们称之为算子损失,定义如下:
\begin{equation}
	\mathcal{L}_{Operator}(\theta)=\frac{1}{NP}\sum_{i=1}^{N}\sum_{j=1}^{P}\left|G_{\theta}(u^{(i)}_0)(x_{u,j}^{(i)},t_{u,j}^{(i)})- u^{(i)}(x_{u,j}^{(i)},t_{u,j}^{(i)})\right|^2 \label{eq:loss_operator}
\end{equation}
第二类是物理模型损失,在PDE的定义域内随机取点,带入方程,计算两边的差异:
\begin{equation}
	\mathcal{L}_{Physics}(\theta)=\frac{1}{NQ}\sum_{i=1}^{N}\sum_{j=1}^{Q}\left|\frac{\partial G_{\theta}(u^{(i)}_0)(x_{r,j}^{(i)},t_{r,j}^{(i)})}{\partial t} - D\frac{\partial^2 G_{\theta}(u^{(i)}_0)(x_{r,j}^{(i)},t_{r,j}^{(i)})}{\partial x^2} - k[G_{\theta}(u^{(i)}_0)(x_{r,j}^{(i)},t_{r,j}^{(i)})]^2\right|^2 \label{eq:loss_physics}
\end{equation}
最终,损失函数为:
\begin{equation}
	\mathcal{L}(\theta)=\mathcal{L}_{Operator}(\theta)+\mathcal{L}_{Physics}(\theta)
\end{equation}

\section{神经网络结构}
\begin{figure}[h]
	\centering
	\subfigure[DeepONet]{
		\includegraphics[width=0.95\linewidth]{deeponet-architecture}
	}
	\caption{DeepONet}
	\label{DeepONet}
\end{figure}
网络结构采用Unstacked DeepONet,这样的好处是代码清晰简洁点。分支网络输入初值函数的离散格式,主干网络输入PDE定义域内的点。将二者的输出做内积,得到网络的预测值;

\section{训练算法}
\begin{enumerate}
	\item 随机生成N个初值函数(本例中N=5000):
	\begin{equation}
		\{u^{(i)}_0(\textbf{x})\}_{i=1}^{N}={[u^{(1)}_0(\textbf{x}),u^{(2)}_0(\textbf{x}),...,u^{(N)}_0(\textbf{x})]}
	\end{equation}
	\item 对于每个初值函数,在其定义域内等距选取m个点(本例中m=100),取这些点对应的函数值作为分支网络的输入:
	\begin{equation}
		\begin{bmatrix}
			u^{(1)}_0(x_1) & u^{(1)}_0(x_2)& ... & u^{(1)}_0(x_m) \\
			u^{(2)}_0(x_1) & u^{(2)}_0(x_2)& ... & u^{(2)}_0(x_m) \\
			\vdots & \vdots & \ddots & \vdots \\
			u^{(N)}_0(x_1) & u^{(N)}_0(x_2) & ... & u^{(N)}_0(x_m)
		\end{bmatrix}
	\end{equation}
	\item 将其送入分支网络,经过前向传播:
	\begin{equation}
		b_k
		\begin{bmatrix}
			u^{(1)}_0(x_1) & u^{(1)}_0(x_2)& ... & u^{(1)}_0(x_m) \\
			u^{(2)}_0(x_1) & u^{(2)}_0(x_2)& ... & u^{(2)}_0(x_m) \\
			\vdots & \vdots & \ddots & \vdots \\
			u^{(N)}_0(x_1) & u^{(N)}_0(x_2) & ... & u^{(N)}_0(x_m)
		\end{bmatrix}
	\end{equation}
	\item 从初边值选择P个点(本例中P=100)
	\begin{equation}
		\textbf{y}_u=y_{u1},y_{u2},...,y_{uP}
	\end{equation}
	\item 将其送入主干网络,经过前向传播:
	\begin{equation}
		t_k(y_{u1},y_{u2},...,y_{uP})
	\end{equation}
	\item 将二者输出做点积,得到近似的算子
	\begin{equation}
		G_\theta(u)(\textbf{y})=\sum_{k=1}^q\underset{Branch}{\underbrace{b_k\left(u(x_1),u(x_2),...,u(x_m)\right)}}.\underset{Trunk}{\underbrace{t_k(\textbf{y})}}
	\end{equation}
	\item 根据公式\ref{eq:loss_operator},计算算子损失;
	\item 从定义域中随机选Q个点
	\begin{equation}
		\textbf{y}_r=y_{r1},y_{r2},...,y_{rQ}
	\end{equation}
	\item 根据公式\ref{eq:loss_physics},计算物理模型损失;
	\item 计算总损失
	\begin{equation}
		\mathcal{L}(\theta)=\mathcal{L}_{operator}(\theta)+\mathcal{L}_{Physics}(\theta)
	\end{equation}
	\item 反向传播,利用梯度下降更新神经网络参数,使损失最小化;
	\item 重复上述过程直至$G_\theta(u_0)(x,t)\approx G(u_0)(x,t)$
\end{enumerate}
{\color{red} 这里需要注意的是,初值函数集合要和训练点集合做卡氏积,即对于任何一个初值函数,它要和所有的训练点分别进行一次网络前馈,并做点积。}

\section{主程序解释}
\subsection{RBF}
高斯径向基函数,是一种高斯核函数,用于计算两个时刻的高斯变量的协方差,用RBF得到的是一个半正定矩阵;
\begin{equation}
	k(x,x') = \sigma^2 exp(-\frac{||x-x'||^2}{2l^2})
\end{equation}

\subsection{gp\_sample}
高斯过程采样函数,对x的定义域[0, 1]做网格划分,得到一列点$x_0, x_1, \cdots , x_n$,记为向量$\mathbf{x}$,作为参数调用RBF函数,得到协方差矩阵,再利用cholesky分解定理得到一列符合正态分布的随机变量$y_0, y_1,\cdots, y_n$,记为$
\mathbf{y}$;以$\mathbf{x}$和$\mathbf{y}$作插值函数,就生成了一个初值函数;方法中通过for循环生成初值函数列表,注意这里返回的就是function对象。

\subsection{ADRICBCDataset}
初边值条件的数据集,主要逻辑再init方法里面。
\begin{enumerate}
	\item 初始化初值训练点\\
	本例中选取100个初值点进行训练,先对高斯过程采样生成的初值函数列表进行离散化,取离散化的函数值作为分支网的输入,注意这里每个函数要和所有的初边值点作用一下,所以要对函数进行复制,复制的次数等于初值点的个数,对[0 ,1]均匀采样100个点作为初值训练点,并调用初值函数得到函数值作为训练的标签值,这里同样需要注意的是,对于每个初值训练点,都要调用所有的初值函数。
	\item 初始化边值训练点\\
	逻辑和上面的初值训练点差不多,区别就是本例采用零边值条件,所以对应的标签值都为0;
\end{enumerate}

\subsection{ADRPhysicsDataset}
物理信息的数据集,逻辑和初边值训练点差不多,区别在于是对$[0, 1] \times [0, 1]$ 均匀采点;

\subsection{ADRNet}
算子网络的具体实现
\subsubsection{init}
激活函数设置为双曲正切(Tanh),损失函数使用均方误差(MSE),主干网络和分支网络根据传入的参数构造多层感知机,就是多层全连接网络。

\subsubsection{forward}
将初值函数经过分支网络前向传播,训练点(x,t)经过主干网络前向传播,对两个网络的输出做内积,得到预测值。

\subsubsection{loss\_icbc}
计算初边值训练损失,将初边值训练点经过网络前馈,计算得到的预测试和标签值的均方误差。

\subsubsection{loss\_physics}
计算物理模型损失,将物理训练点经过网络前馈,对得到的预测值计算各阶导数,按照公式\ref{eq:loss_physics}计算损失。

\subsubsection{loss}
将初边值损失和物理损失进行加总,这里相当于设置二者的权重各位0.5。

\end{document}