\documentclass{article}
\usepackage[a4paper,innermargin=1.2in,outermargin=1.2in,
bottom=1.5in,marginparwidth=1in,marginparsep
=3mm]{geometry}
\usepackage{amsmath,amsthm,amssymb,enumerate}
\usepackage{ctex}
\usepackage{natbib}
\usepackage{graphicx}
\graphicspath{{images/}}
\usepackage{float}
\usepackage{subfigure}
\usepackage{color}
\usepackage{array}
\title{神经网络解Lorenz63方程}
\author{MG21210021李庆春}
%\date{2022.8.20}
\linespread{1.25}
\bibliographystyle{plain}

\usepackage{listings}
\lstset{
	columns=fixed,       
	numbers=left,                                        % 在左侧显示行号
	numberstyle=\tiny\color{black},                       % 设定行号格式
	frame=none,                                          % 不显示背景边框
	backgroundcolor=\color[RGB]{245,245,244},            % 设定背景颜色
	keywordstyle=\color[RGB]{40,40,255},                 % 设定关键字颜色
	numberstyle=\footnotesize\color{black},           
	commentstyle=\it\color[RGB]{0,96,96},                % 设置代码注释的格式
	stringstyle=\rmfamily\slshape\color[RGB]{128,0,0},   % 设置字符串格式
	showstringspaces=false,                              % 不显示字符串中的空格
	language=python,                                        % 设置语言
}


\begin{document}
	
	\maketitle
	
	\section{Lorenz63方程简述}
	\begin{equation}
		\begin{cases}
			\frac{dx}{dt} = \sigma(y -x) \\
			\frac{dy}{dt} = x(\rho - z) - y \\
			\frac{dz}{dt} = xy - \beta z
		\end{cases}
	\end{equation}
	
	\section{非混沌情形}
	$\rho = 15 , \sigma = 10 , \beta = \frac{8}{3}$ 初值:[-8., 7., 27.],RK方法的数值解如图\ref{lorenz63-rk}:
	\begin{figure}[h]
		\centering
		\subfigure[lorenz63 二维视图.]{
			\includegraphics[width=0.45\linewidth]{lorenz63-rk-non-chaos-01}
		}
		\subfigure[lorenz63 三维视图.]{
			\includegraphics[width=0.45\linewidth]{lorenz63-rk-non-chaos-02}
		}
		\caption{lorenz63-RK}
		\label{lorenz63-rk}
	\end{figure}
	
	\subsection{不带观测的神经网络}
	损失函数使用均方误差(MSE),网络结构:[1, 32, 32, 3],初值:[-8, 7, 27],迭代轮数:500000,对训练数据进行正规化,关键代码如下:\\
	\begin{lstlisting}
		scheduler = torch.optim.lr_scheduler.ReduceLROnPlateau(optimizer, 
		min_lr=1e-7, mode='min',factor=0.5,patience=50000,verbose=True)
		self.loss_func = nn.MSELoss(reduction='mean')
		
		layers = [1, 32, 32, 3]
		
		total_points = 300
		x_lb = torch.tensor(0.)
		x_ub = torch.tensor(1.24)
		
		rho = torch.tensor(15.0)
		sigma = torch.tensor(10.0)
		beta = torch.tensor(8.0 / 3.0)
		
		x_train_bc = torch.tensor([[0.]])
		y_train_bc = torch.tensor([[-8., 7., 27.]])
		
		epochs = 500000
	\end{lstlisting} 
	通过调整区间长度发现,最多可训练出[0, 1.24]的结果,误差在$10^{-4}$量级,如图\ref{lorenz63-dnn-non-chaos-[1,1.24]}:
	\begin{figure}[h]
		\centering
		\subfigure[lorenz63 二维视图.]{
			\includegraphics[width=0.45\linewidth]{lorenz63-dnn-non-chaos-01}
		}
		\subfigure[lorenz63 三维视图.]{
			\includegraphics[width=0.45\linewidth]{lorenz63-dnn-non-chaos-02}
		}
		\caption{lorenz63-DNN}
		\label{lorenz63-dnn-non-chaos-[1,1.24]}
	\end{figure}
	
	如果区间较大,则训练不出想要的结果,神经网络会趋于零,如图\ref{lorenz63-dnn-non-chaos-failed-01}展示的是[0, 3]的训练结果:
	\begin{figure}[h]
		\centering
		\subfigure[lorenz63 二维视图.]{
			\includegraphics[width=0.45\linewidth]{lorenz63-dnn-non-chaos-03}
		}
		\subfigure[lorenz63 三维视图.]{
			\includegraphics[width=0.45\linewidth]{lorenz63-dnn-non-chaos-04}
		}
		\caption{lorenz63-DNN}
		\label{lorenz63-dnn-non-chaos-failed-01}
	\end{figure}
	
	\section{混沌情形}
	$\rho = 28 , \sigma = 10 , \beta = \frac{8}{3}$ ,初值[-4., 7., 15.],RK方法的数值解如下图:(暂时没贴)
	\subsection{整数阶Lorenz方程}
	接下来用神经网络训练,离散步长h=0.005,区间[0, 1.5],观测值如下:\\
	\begin{tabular}{| p{1.5cm}<{\centering} || p{1.5cm}<{\centering} | p{1.5cm}<{\centering} | p{1.5cm}<{\centering} | } \hline
		t & x & y & z  \\ \hline \hline
		0.0000 & -4.0000 &   7.0000 &  15.0000 \\ \hline
		0.5000 & 5.6789 &  -2.6593 &  33.0093 \\ \hline
		1.0000 & -12.0143 & -17.4640 &  24.3640 \\ \hline
		1.500 & -1.1838 &  -1.6864 &  15.08598 \\ \hline
	\end{tabular}
	\subsection{分数阶Lorenz方程}
	\subsubsection{Caputo导数及数值逼近}
	Caputo分数阶导数的定义:
	\begin{equation}
		_aD_t^\alpha f(t) = \frac{1}{\Gamma(n - \alpha)}\int_a^t (t - \tau)^{n-\alpha-1} f^{(n)}(\tau) d\tau  \qquad (n-1 \leq \alpha < n).
	\end{equation}
	当a=0时,简记为$D^\alpha$。\\
	分数阶微积分的差商逼近格式可以写成如下统一的形式:
	\begin{equation}
		D^\alpha f(t_n) \approx h^{-\alpha}\sum_{k=0}^{N}c_{n,k}f_k
	\end{equation}
	L1算法(即$0\leq\alpha<1$ ,上式中N=n):
	\begin{equation}
		c_{n,k} = \frac{1}{\Gamma(2-\alpha)}
		\begin{cases}
			-c_{n-1},     & k = 0; \\
			c_{n-k} - c_{n-k-1},  & 1\leq k \leq n-1; \\
			1,   & k=n; \\
			0,   & else.
		\end{cases}
	\end{equation}
	\begin{equation}
		(D_t^\alpha f(t_n))_{L1} = \frac{1}{\Gamma(2 - \alpha)h^\alpha}\sum_{k=0}^{n-1}c_k(f_{n-k} - f_{n-k-1})
	\end{equation}
	其中,$c_l = (l+1)^{1-\alpha} - l^{1-\alpha}$,fPINN论文中另一种等价表述如下:
	
	\begin{equation}
		\begin{aligned}
			&\frac{\partial^\gamma \tilde{u} (\boldsymbol{x}, t)}{\partial t^\gamma} 
			\approx 
			\frac{1}{\Gamma (2 - \gamma)(\Delta t)^\gamma} \\
			&\bigg \{
			-c_{\lceil \lambda t \rceil - 1} \tilde{u} (\boldsymbol{x}, 0) + 
			c_0 \tilde{u} (\boldsymbol{x}, t) + 
			\sum_{k=1}^{\lceil \lambda t \rceil - 1} (c_{\lceil \lambda t \rceil - k} - c_{\lceil \lambda t \rceil - k - 1}) \tilde{u} (\boldsymbol{x}, k\Delta t)
			\bigg \}, \\
			&0 < \gamma < 1
		\end{aligned}
	\end{equation}
	时间步长 $\Delta t = t/\lceil \lambda t \rceil \approx 1/\lambda$,$\lceil \cdot \rceil $ 是向上取整函数,表示离散化后的区间数,相当于L1算法公式中的n;常量因子$\lambda$决定着步长,如$\lambda = 200$,则意味着把单位1长度划分为200个小区间。由于计算机在运算时,存在舍入误差,所以$c_0 \tilde{u} (\boldsymbol{x}, n\Delta t) $写成$c_0 \tilde{u} (\boldsymbol{x}, t) $。\\
	{\color{red} (这里有疑问,t=0的导数是不是没法计算) } \\
	所以,可先对每个训练点计算出其时间划分,然后拼成一个大向量。 \\
	如果不引入观测,训练结果的趋势是趋于0,不符合预期;
	
	\subsubsection{分数阶常微分方程数值解法}
	(直接法)考虑齐次初值条件的分数阶微分方程:
	\begin{equation}
		\begin{cases}
			\frac{\partial^\alpha y(t)}{\partial t^\alpha} = f(t,y(t)), \quad t\in [0,T], \\
			y^{(k)}(0) = 0, \quad k = 0,1,\cdots , m-1.
		\end{cases}
	\end{equation}
	则直接应用分数阶导数的一般差商逼近公式得:
	\begin{equation}
		h^{-\alpha}\sum_{k=0}^{N}c_{n,k}y_k = f(t_n,y_n), \quad n = 0,1,\cdots,[t/h]
	\end{equation}
	上式左边为y在t处的$\alpha$阶导数,注意此时将[0,t]分成了n份,每份长度为h,t对应$t_n$,$y(t_n)$对应$y_n$。则上面方程组可以按下面的方式逐点计算:
	\begin{equation}
		y_N = \frac{h^\alpha}{c_{n,N}}f(t_n,y_n)  - \frac{1}{c_{n,N}}\sum_{k=1}^{N-1}c_{n,k}y_k , \quad n =1,\cdots,[T/h]
	\end{equation}
	其中,N=n(对应到G1算法、D算法、L1算法、线性多步法)或N=n+1(对应到G2算法、L2算法)。上式中,由于$y_0=0$,所以求和项从k=1开始,对于一般的L1算法,可写成如下形式:
	\begin{equation}
		\begin{aligned}
			y_n &= \frac{h^\alpha}{c_{n,n}}f(t_n,y_n)  - \frac{1}{c_{n,n}}\sum_{k=0}^{n-1}c_{n,k}y_k , \quad n =1,\cdots,[T/h] \\
			&= \Gamma(2 - \alpha)h^\alpha f(t_n,y_n) - \sum_{k=1}^{n-1} (c_{n-k}-c_{n-k-1})y_k + c_{n-1}y_0 \\
			&= \Gamma(2 - \alpha)h^\alpha f(t_n,y_n) - \sum_{k=1}^{n-1}[(n-k+1)^{1-\alpha} -2(n-k)^{1-\alpha} + (n-k-1)^{1-\alpha}]y_k + \\
			&\qquad [n^{1-\alpha} - (n-1)^{1-\alpha}]y_0
		\end{aligned}
	\end{equation}
	
	\subsubsection{简单的例子}
	已知正弦、余弦函数的整数阶微分表达式分别为
	\begin{equation}
		\frac{d^k}{dt^k}[\sin at] = a^k \sin(at+\frac{k\pi}{2}), \quad \frac{d^k}{dt^k}[\cos at] = a^k \cos(at+\frac{k\pi}{2})
	\end{equation}
	由Cauchy积分公式可以证明,对于分数阶的微分方程来说,当k为分数时,上述公式仍然成立。所以考虑如下分数阶微分方程:
	\begin{equation}
		\begin{cases}
			\frac{d^\alpha u}{dt^\alpha} = \sin(t+\frac{\alpha\pi}{2}), \qquad 0<\alpha<1 \\
			u(0) = 0
		\end{cases}
	\end{equation}
	(这里暂时不知道解析解,所以作废)
	
	对于$f(t)=t^\lambda, \lambda>-1$,n为大于 $\alpha$的最小整数,则其$\alpha$阶Caputo导数为:
	\begin{equation}
		\begin{aligned}
			D^\alpha t^\lambda &= \frac{1}{\Gamma(n-\alpha)}\int_0^t (t-\tau)^{n-\alpha-1}\frac{d^n \tau^\lambda}{d\tau^n}d\tau \\
			&= \frac{1}{\Gamma(n-\alpha)}\int_0^t (t-\tau)^{n-\alpha-1} \lambda(\lambda-1)\cdots(\lambda-n+1)t^{\lambda-n} d\tau \\
			&= \frac{\Gamma(\lambda+1)}{\Gamma(n-\alpha)\Gamma(\lambda-n+1)}\int_0^t (t-\tau)^{n-\alpha-1} t^{\lambda-n} d\tau \\
			&= \frac{\Gamma(\lambda+1)}{\Gamma(n-\alpha)\Gamma(\lambda-n+1)} t^{\lambda-\alpha}\int_0^1 (1-\tau)^{n-\alpha-1} t^{\lambda-n} d\tau \quad (substitution) \\
			&= \frac{\Gamma(\lambda+1)}{\Gamma(n-\alpha)\Gamma(\lambda-n+1)} t^{\lambda-\alpha} B(n-\alpha, \lambda-n+1) \\
			&= \frac{\Gamma(\lambda+1)}{\Gamma(n-\alpha)\Gamma(\lambda-n+1)} t^{\lambda-\alpha} \frac{\Gamma(n-\alpha)\Gamma(\lambda-n+1)}{\Gamma(\lambda-\alpha+1)}  \\
			&=\frac{\Gamma(\lambda+1)}{\Gamma(\lambda-\alpha+1)} t^{\lambda-\alpha}
		\end{aligned}
	\end{equation}
	这和R-L导数一致,这是由于$D^n t^\lambda$在区间$[0, +\infty)$ 上连续,且$D^kf(0)=0, k=0,1,\cdots,n-1$,所以$D^n$和$D^{-\nu}$可交换。\\
	例子:考虑$y=t^3$,则对应的分数阶微分方程为:
	\begin{equation}
		\begin{cases}
			D^\alpha t^3 = \frac{6}{\Gamma(4-\alpha)} t^{3-\alpha} \\
			y(0) = 0
		\end{cases}
	\end{equation}
	写成隐式的:
	\begin{equation}
		\begin{cases}
			D^\alpha u = \frac{6u}{\Gamma(4-\alpha)} t^{-\alpha} \\
			u(0) = 0.000001
		\end{cases}
	\end{equation}
	数值计算结果如图\ref{ode-fdm}:{\color{red} (这里有疑问,时间是不是一定要从0开始,变成时发现如果不是0作为离散的初始点,则得不到结果) }
	\begin{figure}[h]
		\centering
		\subfigure[显示]{
			\includegraphics[width=0.45\linewidth]{frac-ode-fdm-explicit-01}
		}
		\subfigure[隐式]{
			\includegraphics[width=0.45\linewidth]{frac-ode-fdm-implicit-01}
		}
		\caption{ODE-FDM}
		\label{ode-fdm}
	\end{figure}
	
	\subsubsection{分数阶Lorenz63方程}
	\begin{equation}
		\begin{cases}
			\frac{d^\alpha x}{dt^\alpha} = \sigma(y -x) \\
			\frac{d^\alpha y}{dt^\alpha} = x(\rho - z) - y \\
			\frac{d^\alpha z}{dt^\alpha} = xy - \beta z
		\end{cases}
	\end{equation}
	取$\alpha = 0.99$,离散步长h=0.005,区间[0, 3],隐式收敛阈值$\epsilon = 0.001$,初值[-4., 7., 15.],数值计算结果如图\ref{florenz63-FDM}:
	\begin{figure}[h]
		\centering
		\subfigure[分数阶lorenz63-2D]{
			\includegraphics[width=0.45\linewidth]{florenz63-fdm-2d-01}
		}
		\subfigure[分数阶lorenz63-3D]{
			\includegraphics[width=0.45\linewidth]{florenz63-fdm-3d-01}
		}
		\caption{frac-lorenz63-FDM}
		\label{florenz63-FDM}
	\end{figure}
	
	接下来用神经网络训练,损失函数使用MSE,先用参数$\alpha = 0.99$,离散步长h=0.005,区间[0, 1.5],观测值如下:\\
	\begin{tabular}{| p{1.3cm}<{\centering} || p{1.3cm}<{\centering} | p{1.3cm}<{\centering} | p{1.3cm}<{\centering} | } \hline
		t & x & y & z  \\ \hline \hline
		0.0050 & -3.4659 &  6.7226 & 14.6728 \\ \hline
		0.2500 & 10.4394 & 17.7745 & 17.9601 \\ \hline
		0.5000 & 5.6236 & -0.6432 & 31.2168 \\ \hline
		0.7500 & 0.7111 &  0.8086 & 15.8068 \\ \hline
		1.0000 & 4.2548 &  7.9222 &  9.8927 \\ \hline
		1.2500 & 13.9364 &  7.5091 & 39.7435 \\ \hline
		1.5000 & 0.5546 & -0.6693 & 20.8581 \\ \hline
	\end{tabular} \\
	训练结果如图\ref{florenz63-dnn-01},符合预期:
	\begin{figure}[h]
		\centering
		\subfigure[分数阶lorenz63-2D]{
			\includegraphics[width=0.45\linewidth]{florenz63-dnn-01-2d}
		}
		\subfigure[分数阶lorenz63-3D]{
			\includegraphics[width=0.45\linewidth]{florenz63-dnn-01-3d}
		}
		\caption{frac-lorenz63-FDM}
		\label{florenz63-dnn-01}
	\end{figure}

继续训练[1.5, 3],观测数据如下,注意此时t=0对应的是t=1.5:\\
\begin{tabular}{| p{1.6cm}<{\centering} || p{1.6cm}<{\centering} | p{1.6cm}<{\centering} | p{1.6cm}<{\centering} | } \hline
	t & x & y & z  \\ \hline \hline
	0.0050 & 0.4947 &  -0.6466 &  20.5689 \\ \hline
	0.2500 & -0.7559 &  -1.3882 &  10.706 \\ \hline
	0.5000 & -7.8296 & -14.6883 &  11.2953 \\ \hline
	0.7500 & -7.6905 &   1.1865 &  34.9326 \\ \hline
	1.0000 & 1.0532 &   2.0665 &  17.3576 \\ \hline
	1.2500 & 7.2898 &  12.7107 &  14.2827 \\ \hline
	1.5000 & 9.4016 &   2.1509 &  35.2638 \\ \hline
\end{tabular} \\
训练结果如图\ref{florenz63-dnn-02},除个别点尖锐突变,根据几次训练,突变点在[0.4,0.8]之间,有时是个小尖角,其他部分符合预期:
\begin{figure}[h]
	\centering
	\subfigure[分数阶lorenz63-2D]{
		\includegraphics[width=0.45\linewidth]{florenz63-dnn-02-2d}
	}
	\subfigure[分数阶lorenz63-3D]{
		\includegraphics[width=0.45\linewidth]{florenz63-dnn-02-3d}
	}
	\caption{frac-lorenz63-FDM}
	\label{florenz63-dnn-02}
\end{figure}
所以继续对[0.4,0.8]增加一些观测值,如下:\\
\begin{tabular}{| p{1.6cm}<{\centering} || p{1.6cm}<{\centering} | p{1.6cm}<{\centering} | p{1.6cm}<{\centering} | } \hline
	t & x & y & z  \\ \hline \hline
	0.4000 & -3.0048 &  -5.7771 &   8.0258 \\ \hline
	0.6000 & -15.5885 & -21.0788 &  30.5955 \\ \hline
	0.7000 & -12.4916 &  -3.5173 &  39.8045 \\ \hline
	0.8000 & -3.9029 &   2.2364 &  29.9654 \\ \hline
\end{tabular} \\
得到的训练结果如图\ref{florenz63-dnn-03}:
\begin{figure}[h]
	\centering
	\subfigure[分数阶lorenz63-2D]{
		\includegraphics[width=0.45\linewidth]{florenz63-dnn-03-2d}
	}
	\subfigure[分数阶lorenz63-3D]{
		\includegraphics[width=0.45\linewidth]{florenz63-dnn-03-3d}
	}
	\caption{frac-lorenz63-FDM}
	\label{florenz63-dnn-03}
\end{figure}
继续增加观测值,变为每0.1的长度,引入一个观测值,训练结果如图\ref{florenz63-dnn-04}:
\begin{figure}[h]
	\centering
	\subfigure[分数阶lorenz63-2D]{
		\includegraphics[width=0.45\linewidth]{florenz63-dnn-04-2d}
	}
	\subfigure[分数阶lorenz63-3D]{
		\includegraphics[width=0.45\linewidth]{florenz63-dnn-04-3d}
	}
	\caption{frac-lorenz63-FDM}
	\label{florenz63-dnn-04}
\end{figure}

\section{Burgers方程}
\begin{equation}
	\begin{cases}
		\frac{\partial u}{\partial t} + u \frac{\partial u}{\partial x} = \nu \frac{\partial^2 u}{\partial x^2}  \\
		x \in [-1, 1] \\
		t \in [0, 1]
	\end{cases}
\end{equation}

\section{TODO}
lorenz63时间长的时候,趋于0,什么原因?\\
pinn解PDE的论文,多看看,多总结,作为自己毕设素材 \\
GL算法实现 \\
神经网络解burgers \\

lorenz63-dnn-obs-01  [0,3] 训练 每个0.02 一个观测值,观测值由RK法计算,初值为:[1.508870, -1.531271, 25.46091] ,训练不收敛

lorenz63-dnn-obs-02 [0,1.5]
lorenz63-dnn-obs-03 [0,3] 观测值为iobsdisturb.txt , 导师发是数据;

lorenz63-dnn-12 [0,3]  尝试分段mse,意图为规避训练区间加长,后续趋于0 ,失败,仍然趋于0。换个思路, 把三个导数的平方和相加,取导数加入损失函数。失败

lorenz63-dnn-13 使用L1Smooth + 无穷范数,失败

\end{document}